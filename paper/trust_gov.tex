% Options for packages loaded elsewhere
\PassOptionsToPackage{unicode}{hyperref}
\PassOptionsToPackage{hyphens}{url}
\PassOptionsToPackage{dvipsnames,svgnames,x11names}{xcolor}
%
\documentclass[
  12pt,
]{article}
\usepackage{amsmath,amssymb}
\usepackage{setspace}
\usepackage{iftex}
\ifPDFTeX
  \usepackage[T1]{fontenc}
  \usepackage[utf8]{inputenc}
  \usepackage{textcomp} % provide euro and other symbols
\else % if luatex or xetex
  \usepackage{unicode-math} % this also loads fontspec
  \defaultfontfeatures{Scale=MatchLowercase}
  \defaultfontfeatures[\rmfamily]{Ligatures=TeX,Scale=1}
\fi
\usepackage{lmodern}
\ifPDFTeX\else
  % xetex/luatex font selection
\fi
% Use upquote if available, for straight quotes in verbatim environments
\IfFileExists{upquote.sty}{\usepackage{upquote}}{}
\IfFileExists{microtype.sty}{% use microtype if available
  \usepackage[]{microtype}
  \UseMicrotypeSet[protrusion]{basicmath} % disable protrusion for tt fonts
}{}
\usepackage{xcolor}
\usepackage[margin=1in]{geometry}
\usepackage{longtable,booktabs,array}
\usepackage{calc} % for calculating minipage widths
% Correct order of tables after \paragraph or \subparagraph
\usepackage{etoolbox}
\makeatletter
\patchcmd\longtable{\par}{\if@noskipsec\mbox{}\fi\par}{}{}
\makeatother
% Allow footnotes in longtable head/foot
\IfFileExists{footnotehyper.sty}{\usepackage{footnotehyper}}{\usepackage{footnote}}
\makesavenoteenv{longtable}
\usepackage{graphicx}
\makeatletter
\def\maxwidth{\ifdim\Gin@nat@width>\linewidth\linewidth\else\Gin@nat@width\fi}
\def\maxheight{\ifdim\Gin@nat@height>\textheight\textheight\else\Gin@nat@height\fi}
\makeatother
% Scale images if necessary, so that they will not overflow the page
% margins by default, and it is still possible to overwrite the defaults
% using explicit options in \includegraphics[width, height, ...]{}
\setkeys{Gin}{width=\maxwidth,height=\maxheight,keepaspectratio}
% Set default figure placement to htbp
\makeatletter
\def\fps@figure{htbp}
\makeatother
\setlength{\emergencystretch}{3em} % prevent overfull lines
\providecommand{\tightlist}{%
  \setlength{\itemsep}{0pt}\setlength{\parskip}{0pt}}
\setcounter{secnumdepth}{-\maxdimen} % remove section numbering
% definitions for citeproc citations
\NewDocumentCommand\citeproctext{}{}
\NewDocumentCommand\citeproc{mm}{%
  \begingroup\def\citeproctext{#2}\cite{#1}\endgroup}
\makeatletter
 % allow citations to break across lines
 \let\@cite@ofmt\@firstofone
 % avoid brackets around text for \cite:
 \def\@biblabel#1{}
 \def\@cite#1#2{{#1\if@tempswa , #2\fi}}
\makeatother
\newlength{\cslhangindent}
\setlength{\cslhangindent}{1.5em}
\newlength{\csllabelwidth}
\setlength{\csllabelwidth}{3em}
\newenvironment{CSLReferences}[2] % #1 hanging-indent, #2 entry-spacing
 {\begin{list}{}{%
  \setlength{\itemindent}{0pt}
  \setlength{\leftmargin}{0pt}
  \setlength{\parsep}{0pt}
  % turn on hanging indent if param 1 is 1
  \ifodd #1
   \setlength{\leftmargin}{\cslhangindent}
   \setlength{\itemindent}{-1\cslhangindent}
  \fi
  % set entry spacing
  \setlength{\itemsep}{#2\baselineskip}}}
 {\end{list}}
\usepackage{calc}
\newcommand{\CSLBlock}[1]{\hfill\break\parbox[t]{\linewidth}{\strut\ignorespaces#1\strut}}
\newcommand{\CSLLeftMargin}[1]{\parbox[t]{\csllabelwidth}{\strut#1\strut}}
\newcommand{\CSLRightInline}[1]{\parbox[t]{\linewidth - \csllabelwidth}{\strut#1\strut}}
\newcommand{\CSLIndent}[1]{\hspace{\cslhangindent}#1}
\usepackage{array}
\usepackage{caption}
\usepackage{graphicx}
\usepackage{siunitx}
\usepackage{colortbl}
\usepackage{multirow}
\usepackage{hhline}
\usepackage{calc}
\usepackage{tabularx}
\usepackage{threeparttable}
\usepackage{wrapfig}
\usepackage{fullpage}
\usepackage{lscape}
\newcommand{\blandscape}{\begin{landscape}}
\newcommand{\elandscape}{\end{landscape}}
\usepackage{titlesec}
\titleformat*{\section}{\normalsize\bfseries}
\titleformat*{\subsection}{\normalsize\itshape}
\usepackage{titling}
\usepackage{booktabs}
\usepackage{longtable}
\usepackage{array}
\usepackage{multirow}
\usepackage{wrapfig}
\usepackage{float}
\usepackage{colortbl}
\usepackage{pdflscape}
\usepackage{tabu}
\usepackage{threeparttable}
\usepackage{threeparttablex}
\usepackage[normalem]{ulem}
\usepackage{makecell}
\usepackage{xcolor}
\ifLuaTeX
  \usepackage{selnolig}  % disable illegal ligatures
\fi
\usepackage{bookmark}
\IfFileExists{xurl.sty}{\usepackage{xurl}}{} % add URL line breaks if available
\urlstyle{same}
\hypersetup{
  pdftitle={Comparative Estimates of Public Trust in Government Across 115 Countries, 1973--2020},
  colorlinks=true,
  linkcolor={black},
  filecolor={Maroon},
  citecolor={black},
  urlcolor={Blue},
  pdfcreator={LaTeX via pandoc}}

\title{Comparative Estimates of Public Trust in Government Across 115 Countries, 1973--2020}
\author{}
\date{\vspace{-2.5em}}

\begin{document}
\maketitle

\setstretch{1.5}
\pagenumbering{gobble}

\section*{Authors}\label{authors}
\addcontentsline{toc}{section}{Authors}

\begin{itemize}
\tightlist
\item
  Yuehong Cassandra Tai, corresponding author, ORCID: \url{https://orcid.org/0000-0001-7303-7443}, Assistant Research Professor, Center for Social Data Analytics, Pennsylvania State University, \href{mailto:yhcasstai@psu.edu}{\nolinkurl{yhcasstai@psu.edu}}
\end{itemize}

\section{Data and Code Availability}\label{data-and-code-availability}

The code used to generate the dataset and conduct validation test are openly available at: \url{https://github.com/Tyhcass/TGOV}.

\section{Conflict of Interest Disclosure}\label{conflict-of-interest-disclosure}

The author declares no competing interests.

\section{Acknowledgements}\label{acknowledgements}

\pagebreak

\renewcommand{\baselinestretch}{1}
\selectfont
\maketitle
\renewcommand{\baselinestretch}{1.5}
\selectfont

\begin{abstract}
Political trust plays a critical role in understanding key political phenomena, including regime support, democratic legitimacy, policy preferences, and political behavior. However, the lack of comparable, cross-national data has limited scholars' ability to analyze the relationship of political trust with quantities of interest and to generalize findings across different countries and time periods. To address this gap, this paper introduces the Trust in Government (TGOV) Dataset—a time-series, cross-sectional resource covering 115 countries/territories from 1973 to 2020, constructed using a Bayesian latent variable model on 1,555 country-year observations from 189 national and cross-national surveys. The TGOV dataset has been validated through a series of internal, external, and construct validation tests. It enables scholars to investigate the dynamic relationship between political trust and institutional performance, policy outcomes, and crisis resilience across diverse political systems.

\end{abstract}

Keywords: trust in government, latent variable model, election, corruption, approval ratings

\pagebreak

\pagenumbering{arabic}

\section{Introduction}\label{introduction}

Trust is critical for understanding key political, societal, and economic issues, including but not limited to regime support, democratic legitimacy, and public confidence in elections (Easton 1975; Norris et al. 2002; Wuttke, Schimpf, and Schoen 2020; Kerr, King, and Wahman 2024); civic duty, participation, and law compliance (Valgarsson et al. 2021; Hooghe and Marien 2013; Tyler 1990); government performance, policy preferences, and inequality (Rose and Mishler 2010; Goubin and Hooghe 2020; Rudolph and Evans 2005); as well as broader outcomes such as public health responses to crises like the COVID-19 pandemic (Zaki et al. 2022; Devine et al. 2021).
However, regardless of universal exist of political trust and these issues across different regimes, a significant challenge for scholars examining these issues or testing theories generally is the limited availability of comprehensive comparative data on trust in government across countries and over time.
Existing datasets often suffer from fragmented coverage, restricting researchers' ability to rigorously compare trust dynamics across diverse political and regional contexts (Devine 2024; Kerr, King, and Wahman 2024).
Most of these studies focus on a single country, dominated by studies in the U.S. and U.K. and even when comparative research was studied, most of them focused on certain region or democratic countries, such European countries (Devine 2024).
The lack of time-series comparative trust data makes it impossible to examine the theories about the relationship between temporal changes of trust and other quantities of interest, most of which have temporal features.

To address this critical gap, I introduce a time-series cross-national dataset measuring public trust in national governments (TGOV), covering 115 countries from 1973 to 2020.
Using a sophisticated Bayesian latent variable model developed by Solt (2020b), this dataset synthesizes various survey sources into robust, comparable trust estimates, drawn from 189 with 2,136 country-year-item covering 47 years.
In addition to providing mean estimates of trust, the dataset includes full posterior samples, enabling scholars to explicitly incorporate the measurement uncertainty inherent in latent variable models.
By offering temporal and cross-national coverage, the TGOV dataset facilitates comparative research on the causes and consequences of political trust, enhancing our understanding of democratic governance, electoral behavior, policy development, and related areas.

\section{Data \& Methods}\label{data-methods}

\begin{figure}
\centering
\includegraphics{trust_gov_files/figure-latex/itemcountryplots-1.pdf}
\caption{\label{fig:itemcountryplots}Countries and Years with the Most Observations in the Source Data of Trust in Government \label{tgov_item_country_plots}}
\end{figure}

\subsection{Raw Data}\label{raw-data}

Although many national and cross-national surveys have asked questions on trust toward national government, comparative data at the aggregate level is sparse and fragmented.
This fragmentation is primarily due to limited data availability across countries and years, as well as inconsistencies in question wording and interpretation.

To construct a dynamic and comparable trust in government dataset, I systematically reviewed 189 unique survey projects spanning 116 countries over 47 year.
I identified 10 unique survey questions that captured public attitudes toward trust in national governments.
To improve comparability and reduce uncertainty from sparse data, I followed standard practice by excluding survey items that were rarely asked (Woo, Goldberg, and Solt 2023).

Compared to other important survey questions on public gender egalitarianism (Woo, Goldberg, and Solt 2023) and gay rights (Woo et al. Forthcoming), data on political trust in government is relative rich.
In the country-years span, among the 2,674 country-years, 58\% of it has available information.
However, if we have observations for every year in each country surveyed, the number would be 5,452.
In fact, even collecting as many available national and cross-national data as possible, the current source data has 1,555 country-years which are 29\% of a complete set of total country-year.

The left panel of Figure \nobreakspace{}\ref{tgov_item_country_plots} maps the global distribution of observed country-years.
European and Latin American countries have longer time series due to the frequent administration of Eurobarometer and AmericasBarometer surveys and strong scholarly interest in democratic and emerging regimes.
In contrast, data from Asian and African countries is limited.
The upper right panel further illustrates this geographical disparity: Germany leads with 54 country-year-item observations, followed by Spain and Finland.
The lower right panel shows how few relevant survey items existed before 1990.
Country coverage reached its peak in 2018, when respondents in 111 countries were asked items about trust in government.
Overall, although questions on trust in government appeared as early as 1970s, they were not surveyed steadily and broadly until the 1990s, with geographic disparity.

In the next section, I describe how I leveraged this sparse and incomparable survey data to generate complete, comparable time-series trust scores using a latent variable model.

\subsection{Measurements}\label{measurements}

Latent variable measurement assumes the concept of interest is not directly observable but can be inferred from individuals' responses to relevant questions.
Recently, pioneering studies have developed latent-variable models specifically tailored to cross-national survey data (see Claassen 2019; Caughey, O'Grady, and Warshaw 2019; McGann, Dellepiane-Avellaneda, and Bartle 2019; Kolczynska et al. 2020).
In this paper, I adopt the Dynamic Comparative Public Opinion (DCPO) model developed by Solt (2020b), which provides a better fit to survey data compared to alternative models (Claassen 2019; Caughey, O'Grady, and Warshaw 2019), and effectively manages data sparsity without requiring dense survey coverage or additional population characteristics (McGann, Dellepiane-Avellaneda, and Bartle 2019; Kolczynska et al. 2020).

The DCPO model can address the two principal challenges posed by the source data: incomparability and sparsity.
To tackle the incomparability of different survey questions, the model includes two parameters:
the \emph{difficulty} parameter, which captures how much trust is required to respond affirmatively to each survey question (e.g., ``a great deal'' vs.~``somewhat'' trust), and the \emph{dispersion} parameter, which indicates how sensitively survey responses reflect changes in the latent trust variable.
A lower dispersion score means that a small change in responses corresponds closely to a substantial shift in the latent trust level.
Together, the difficulty and dispersion parameters allow the generation of comparable trust estimates across different surveys and questions.
For data sparsity, the model adopts random-walk priors, estimating missing latent values as the previous year's estimate plus a random shock. This approach smooths estimates over time, even in years with limited data, though with greater measurement uncertainty.
For more detailed information about the DCPO model, see Solt (2020b, 3--8).

Trust in government is estimated in DCPO model through the \texttt{DCPO} package for R (Solt 2020a).
The estimates in all 2,674 country-years spanned by the source data are the TGOV scores.

\section{Results}\label{results}

\begin{figure}
\centering
\includegraphics{trust_gov_files/figure-latex/cs-1.pdf}
\caption{\label{fig:cs}TGOV Scores, Most Recent Available Year \label{cs_mry}}
\end{figure}

Figure\nobreakspace{}\ref{cs_mry} displays the most recent available TGOV score for each of the 115 countries and territories in the dataset.
China and several Central Asian countries dominate the top positions, aligning with previous research indicating high levels of trust in these governments (Schneider 2017; Paturyan and Melkonyan 2024; Byaro and Kinyondo 2020).
Although respondents' understanding about political trust may differ across regime types, prior research has shown that trust in central political institutions can be comparably measured using latent variable models (Schneider 2017).
Less corrupt countries like Denmark and Switzerland rank highly, while Libya, Tunisia, Venezuela, Iraq, and Brazil show the lowest recent trust scores. These lower-ranked countries faced serious challenges around the time of measurement, including corruption (Venezuela, Iraq), election-related violence (Brazil), and conflict or security threats (Tunisia, Libya).

\begin{figure}
\centering
\includegraphics{trust_gov_files/figure-latex/ts-1.pdf}
\caption{\label{fig:ts}TGOV Scores Over Time Within Selected Countries \label{ts_plots}}
\end{figure}

Figure\nobreakspace{}\ref{ts_plots} illustrates changes in TGOV scores over time for 12 selected countries.
The dataset's extensive geographic coverage enables comparative analyses of regions and countries often overlooked in prior research (see Wilson and Knutsen 2022).

The different trends are observed across these countries.
For example, public trust in government has risen prominently in countries such as Germany, India, the Philippines, and Nigeria---likely due to stable governance under Merkel in Germany and Modi in India (OECD 2025; Sardesai and Shastri 2023), populist administration policies in the Philippines (Curato 2017), and decreasing violence levels in Nigeria (Harding and Nwokolo 2024).
In contrast, trust levels have remained consistently high in China (Tang 2016) and relatively low in Australia (Stoker, Evans, and Halupka 2018).

In contrast, TGOV scores have steadily or dramatically declined in Greece, Mexico, Argentina, and the United States, primarily due to economic crises in Greece (Ervasti, Kouvo, and Venetoklis 2019), widespread corruption in Mexico (Morris and Klesner 2010), financial instability and political dysfunction in Argentina (Council on Foreign Relations 2024), and rising political polarization and partisanship in the United States (Hetherington and Rudolph 2018).

Some countries exhibit fluctuations, as seen in the United Kingdom and Turkey.
In the UK, fluctuations could be associated with Brexit, sovereignty debates, and immigration issues (Guardian 2025).
In Turkey, variations may reflect the personalization of political power and economic volatility (Pew Research Center 2024).

The variations across countries and over time within countries in TGOV scores provides valuable opportunities for scholarly analysis.
To ensure that the validity of TGOV scores across diverse contexts and over time, I conducted a series of convergent and construct validation tests, following standard practices for cross-national public opinion research (Hu et al. 2024)

\begin{figure}
\centering
\includegraphics{trust_gov_files/figure-latex/internalval-1.pdf}
\caption{\label{fig:internalval}Convergent Validation Through Individual TGOV Source Data Survey Items \label{inter_val}}
\end{figure}

Convergent validation tests whether a measure is empirically associated with alternative indicators of the same concept (Adcock and Collier 2001, 540).
Specifically, I conducted `internal' convergent validation tests (see, e.g., Caughey, O'Grady, and Warshaw 2019, 689; Solt 2020b, 10) by comparing TGOV scores against individual items from the source data that were used to generate them.

Figure\nobreakspace{}\ref{inter_val} presents three validation plots comparing TGOV scores with the percentage of respondents expressing at least some trust---calculated using responses equal to or above the median value of each scale.
The left panel shows a scatterplot of country-years in which the TGOV scores are plotted against responses to the Eurobarometer question: ``Please tell me how much you personally trust each of the following institutions using a scale from 1 to 10, where {[}1{]} means `you do not trust the institution at all' and {[}10{]} means `you trust it completely'.''
The strong correlation (R = 0.87) indicates that TGOV scores effectively capture variations in trust across country-years.

The middle panel compares TGOV score with the responses to the question: ``How much do you trust the national government in your country?'' from the Wellcome Global Monitor Survey in 2018.
This item was asked in more countries than any other trust question in a single survey over the past decade, and the strong correlation (R = 0.94) demonstrates the broad applicability of TGOV scores across diverse contexts.

Finally, the right panel compares the trend of the longest item used since 1984 in the Germany Allbus survey: ``How much trust do you place in the federal government?''
The TGOV scores align with the trend over time, effectively capturing historical changes.
In all tests, the correlations are evaluated by incorporating uncertainty in the measures.

\begin{figure}
\centering
\includegraphics{trust_gov_files/figure-latex/tgovextval-1.pdf}
\caption{\label{fig:tgovextval}External Validation Using Trust Data \label{tgov_ev1}}
\end{figure}

I then conducted three `external' convergent validation tests, using survey items that were not part of the TGOV score estimation but are theoretically related to trust in government.
These three items include trust in elections, parliament, and public administration in their respective countries.

Figure\nobreakspace{}\ref{tgov_ev1} displays the results of this group of validation tests.
I plot the TGOV score against public trust in elections from the available data in the World Value Survey Wave 7\footnote{The current data in this analysis is based on surveys conducted before 2021. The future iteration will update more availabl data.}, which asked respondents how much confidence they have in elections in the left plot.
The TGOV score is then compared against public confidence in parliament from the European Quality of Life Survey (EQLS) in the center plot and against the percentage of respondents who expressed trust in public administration in their country, as measured by Eurobarometer, in the right panel.

Across all three tests, the TGOV measure was positively correlated with the three types of public trust, with the strongest correlation observed with trust in parliament{[}R = 0.82{]}, and moderate to strong correlations with trust in public administration {[}R = 0.74{]} and elections {[}R = 0.8{]}.

\begin{figure}
\centering
\includegraphics{trust_gov_files/figure-latex/tgovextval2-1.pdf}
\caption{\label{fig:tgovextval2}Construct Validation Using Perceived Corruption, Satisfaction, and Approval Rating Data \label{tgov_ev2}}
\end{figure}

Construct validation, which assesses whether a measure empirically correlates with other indicators that are theoretically expected to be causally related (Adcock and Collier 2001, 542), was also conducted.
I focused on three such indicators: the public's satisfaction with political system performance, the public's perception of corruption, and executive approval ratings.
Abundant research has shown that trust is a strong predictor of satisfaction with political system performance (Hetherington and Rudolph 2018; Chanley, Rudolph, and Rahn 2000) and approval ratings (Citrin and Luks 2001; Miller and Borrelli 1991), while also being a negative consequence of perceived corruption (Anderson and Tverdova 2003).

The results are presented in Figure\nobreakspace{}\ref{tgov_ev2}.
The left panel shows a clear positive relationship between TGOV scores and satisfaction with political system performance, measured as the percentage of individuals expressing at least some satisfaction with political system performance in the WVS Wave 7.

A similar positive correlation between TGOV scores and executive approval ratings appears in the right panel.
The approval ratings are drawn from smoothed estimates for OECD countries in 2018 from the Executive Approval Project (version 2) (Carlin Ryan et al. 2019).

The center panel shows a negative relationship between TGOV scores and perceptions of widespread corruption, as surveyed in Eurobarometer, the Comparative Study of Electoral Systems (CSES), and the WVS in year of 2017.
In all three tests, the direction of correlation aligns with theoretical expectations.

In sum, all these validations provide strong evidence that the TGOV scores are a valid measure of public trust in national government.

\section{Discussion \& Conclusion}\label{discussion-conclusion}

Although political trust is a long-standing interdisciplinary topic studied by economists, political scientists, sociologists, and psychologists among others, its importance increases during governance crises, including but not limited to public health emergencies, climate change, and rising polarization and populism.
However, our understanding of political trust has been limited to single countries or regions with rich longitudinal data, which may not generalize to other areas, or to snapshots of cross-sectional analysis that cannot capture changes over time (Kolczynska et al. 2020; Devine 2024).
The TGOV score allows scholars from various disciplines to explore the causes and consequences of trust in government across countries over time.

To address missing data at the country-year level, random-walk priors were used in the DCPO model, as described in the methodology.
This approach smooths the estimates, making time-series data possible, but it also introduces greater measurement uncertainty.
Ignoring this uncertainty will distort both statistical and substantial inferences, as demonstrated in Tai, Hu, and Solt (2024).
Therefore, I suggest that scholars using the TGOV data in their analyses incorporate measurement uncertainty into their models.
There are several ways to account for uncertainty (see, e.g., Tai, Hu, and Solt 2024; Woo et al. Forthcoming), and to facilitate this process, the entire dataset from four chains has been provided in the dataverse, in addition to the mean measure of trust in government.

The TGOV dataset will be updated regularly to include the most recent publicly available data, offering invaluable opportunities for advancing the study of politics, governance, and state-society relations.

\section*{References}\label{references}

\phantomsection\label{refs-text}
\begin{CSLReferences}{1}{0}
\bibitem[\citeproctext]{ref-Adcock2001}
Adcock, Robert, and David Collier. 2001. {``Measurement Validity: A Shared Standard for Qualitative and Quantitative Research.''} \emph{American Political Science Review} 95 (3): 529--46.

\bibitem[\citeproctext]{ref-Anderson2003}
Anderson, Christopher J, and Yuliya V Tverdova. 2003. {``Corruption, Political Allegiances, and Attitudes Toward Government in Contemporary Democracies.''} \emph{American Journal of Political Science} 47 (1): 91--109.

\bibitem[\citeproctext]{ref-Byaro2020}
Byaro, Mwoya, and Abel Kinyondo. 2020. {``Citizens' Trust in Government and Their Greater Willingness to Pay Taxes in Tanzania: A Case Study of Mtwara, Lindi, and Dar Es Salaam Regions.''} \emph{Poverty \& Public Policy} 12 (1): 73--83.

\bibitem[\citeproctext]{ref-carlin2019executive}
Carlin Ryan, E, Hartlyn Jonathan, Hellwig Timothy, J Love Gregory, Martinez-Gallardo Cecilia, and M Singer Matthew. 2019. {``Executive Approval Database 2.0.''}

\bibitem[\citeproctext]{ref-Caughey2019}
Caughey, Devin, Tom O'Grady, and Christopher Warshaw. 2019. {``Policy Ideology in European Mass Publics, 1981--2016.''} \emph{American Political Science Review} 113 (3): 674--93.

\bibitem[\citeproctext]{ref-chanley_origins_2000}
Chanley, Virginia A., Thomas J. Rudolph, and Wendy M. Rahn. 2000. {``The {Origins} and {Consequences} of {Public} {Trust} in {Government}: {A} {Time} {Series} {Analysis}.''} \emph{Public Opinion Quarterly} 64 (3): 239--56.

\bibitem[\citeproctext]{ref-citrin2001political}
Citrin, Jack, and Samantha Luks. 2001. {``Political Trust Revisited: D{é}j{à} Vu All over Again?''} \emph{What Is It about Government That Americans Dislike}, 9--27.

\bibitem[\citeproctext]{ref-Claassen2019}
Claassen, Christopher. 2019. {``Estimating Smooth Country--Year Panels of Public Opinion.''} \emph{Political Analysis} 27 (1): 1--20.

\bibitem[\citeproctext]{ref-cfr_argentina}
Council on Foreign Relations. 2024. {``Argentina's Struggle for Stability.''} \url{https://www.cfr.org/backgrounder/argentinas-struggle-stability}.

\bibitem[\citeproctext]{ref-Curato2017}
Curato, Nicole. 2017. {``Flirting with Authoritarian Fantasies? Rodrigo Duterte and the New Terms of Philippine Populism.''} \emph{Journal of Contemporary Asia} 47 (1): 142--53.

\bibitem[\citeproctext]{ref-Devine2024}
Devine, Daniel. 2024. {``Does Political Trust Matter? A Meta-Analysis on the Consequences of Trust.''} \emph{Political Behavior} 46 (4): 2241--62.

\bibitem[\citeproctext]{ref-Devine2021trust}
Devine, Daniel, Jennifer Gaskell, Will Jennings, and Gerry Stoker. 2021. {``Trust and the Coronavirus Pandemic: What Are the Consequences of and for Trust? An Early Review of the Literature.''} \emph{Political Studies Review} 19 (2): 274--85.

\bibitem[\citeproctext]{ref-Easton1975}
Easton, David. 1975. {``A Re-Assessment of the Concept of Political Support.''} \emph{British Journal of Political Science} 5 (4): 435--57.

\bibitem[\citeproctext]{ref-Ervasti2019}
Ervasti, Heikki, Antti Kouvo, and Takis Venetoklis. 2019. {``Social and Institutional Trust in Times of Crisis: Greece, 2002--2011.''} \emph{Social Indicators Research} 141: 1207--31.

\bibitem[\citeproctext]{ref-Goubin2020}
Goubin, Silke, and Marc Hooghe. 2020. {``The Effect of Inequality on the Relation Between Socioeconomic Stratification and Political Trust in Europe.''} \emph{Social Justice Research}, 1--29.

\bibitem[\citeproctext]{ref-guardian_covid2025}
Guardian, The. 2025. {``How COVID Changed the Way We Think.''} 2025. \url{https://www.theguardian.com/uk-news/2025/mar/17/how-covid-changed-the-way-we-think}.

\bibitem[\citeproctext]{ref-Harding2024}
Harding, Robin, and Arinze Nwokolo. 2024. {``Terrorism, Trust, and Identity: Evidence from a Natural Experiment in Nigeria.''} \emph{American Journal of Political Science} 68 (3): 942--57.

\bibitem[\citeproctext]{ref-Hetherington2018}
Hetherington, Marc J, and Thomas J Rudolph. 2018. {``Political Trust and Polarization.''} \emph{The Oxford Handbook of Social and Political Trust}, March.

\bibitem[\citeproctext]{ref-Hooghe2013}
Hooghe, Marc, and Sofie Marien. 2013. {``A Comparative Analysis of the Relation Between Political Trust and Forms of Political Participation in Europe.''} \emph{European Societies} 15 (1): 131--52.

\bibitem[\citeproctext]{ref-Hu2023}
Hu, Yue, Yuehong Cassandra Tai, Hyein Ko, Byung-Deuk Woo, and Frederick Solt. 2024. {``An Incomplete Recipe: One-Dimensional Latent Variables Do Not Capture the Full Flavor of Democratic Support.''} SocArXiv. \url{https://osf.io/preprints/socarxiv/rym8g/}.

\bibitem[\citeproctext]{ref-Kerr2024}
Kerr, Nicholas, Bridgett A King, and Michael Wahman. 2024. {``The Global Crisis of Trust in Elections.''} \emph{Public Opinion Quarterly}. Oxford University Press.

\bibitem[\citeproctext]{ref-Kolczynska2020}
Kolczynska, Marta, Paul-Christian Bürkner, Lauren Kennedy, and Aki Vehtari. 2020. {``Trust in State Institutions in Europe, 1989-2019.''}

\bibitem[\citeproctext]{ref-McGann2019}
McGann, Anthony, Sebastian Dellepiane-Avellaneda, and John Bartle. 2019. {``Parallel Lines? Policy Mood in a Plurinational Democracy.''} \emph{Electoral Studies} 58: 48--57.

\bibitem[\citeproctext]{ref-miller1991confidence}
Miller, Arthur H, and Stephen A Borrelli. 1991. {``Confidence in Government During the 1980s.''} \emph{American Politics Quarterly} 19 (2): 147--73.

\bibitem[\citeproctext]{ref-Morris2010}
Morris, Stephen D, and Joseph L Klesner. 2010. {``Corruption and Trust: Theoretical Considerations and Evidence from Mexico.''} \emph{Comparative Political Studies} 43 (10): 1258--85.

\bibitem[\citeproctext]{ref-Norris2002}
Norris, Pippa et al. 2002. \emph{Democratic Phoenix: Reinventing Political Activism}. Cambridge University Press.

\bibitem[\citeproctext]{ref-Oecd_trust2025}
OECD. 2025. {``Trust in Government.''} \url{https://www.oecd.org/en/data/indicators/trust-in-government.html}.

\bibitem[\citeproctext]{ref-Paturyan2024}
Paturyan, Yevgenya Jenny, and Sara Melkonyan. 2024. {``Revolution, Covid-19, and War in Armenia: Impacts on Various Forms of Trust.''} \emph{Caucasus Survey} 1 (aop): 1--28.

\bibitem[\citeproctext]{ref-Pewturks2024}
Pew Research Center. 2024. {``Turks Lean Negative on Erdoğan, Give National Government Mixed Ratings.''} 2024. \url{https://www.pewresearch.org/global/2024/10/16/turks-lean-negative-on-erdogan-give-national-government-mixed-ratings/}.

\bibitem[\citeproctext]{ref-Rose2010}
Rose, Richard, and William Mishler. 2010. {``The Impact of Macro-Economic Shock on Russians.''} \emph{Post-Soviet Affairs} 26 (1): 38--57.

\bibitem[\citeproctext]{ref-Rudolph2005}
Rudolph, Thomas J, and Jillian Evans. 2005. {``Political Trust, Ideology, and Public Support for Government Spending.''} \emph{American Journal of Political Science} 49 (3): 660--71.

\bibitem[\citeproctext]{ref-Sardesai2023}
Sardesai, Shreyas, and Sandeep Shastri. 2023. {``Intensity of Trust in Institutions in India: The Emerging Paradox.''} \emph{How Asians View Democratic Legitimacy}, 209.

\bibitem[\citeproctext]{ref-Schneider2017}
Schneider, Irena. 2017. {``Can We Trust Measures of Political Trust? Assessing Measurement Equivalence in Diverse Regime Types.''} \emph{Social Indicators Research} 133 (3): 963--84.

\bibitem[\citeproctext]{ref-Solt2020a}
Solt, Frederick. 2020a. {``{DCPO}: Dynamic Comparative Public Opinion.''} Available at the Comprehensive R Archive Network (CRAN). \texttt{https://CRAN.R-project.org/package=DCPO}.

\bibitem[\citeproctext]{ref-Solt2020c}
---------. 2020b. {``Modeling Dynamic Comparative Public Opinion.''} SocArXiv. \texttt{https://osf.io/\ preprints/socarxiv/d5n9p}.

\bibitem[\citeproctext]{ref-Stoker2018}
Stoker, Gerry, Mark Evans, and Max Halupka. 2018. {``Trust and Democracy in Australia: Democratic Decline and Renewal.''}

\bibitem[\citeproctext]{ref-Tai2024}
Tai, Yuehong Cassandra, Yue Hu, and Frederick Solt. 2024. {``Democracy, Public Support, and Measurement Uncertainty.''} \emph{American Political Science Review} 118 (1): 512--18.

\bibitem[\citeproctext]{ref-Tang2016}
Tang, Wenfang. 2016. \emph{Populist Authoritarianism: Chinese Political Culture and Regime Sustainability}. Oxford University Press.

\bibitem[\citeproctext]{ref-tyler1990justice}
Tyler, Tom R. 1990. {``Justice, Self-Interest, and the Legitimacy of Legal and Political Authority.''}

\bibitem[\citeproctext]{ref-Valgardhsson2021}
Valgarsson, VO, G Stoker, D Devine, J Gaskell, and W Jennings. 2021. {``Disengagement and Political Trust: Divergent Pathways.''} \emph{Oxford Handbook of Political Participation}. Oxford University Press.

\bibitem[\citeproctext]{ref-Wilson2021}
Wilson, Matthew Charles, and Carl Henrik Knutsen. 2022. {``Geographical Coverage in Political Science Research.''} \emph{Perspectives on Politics} 20 (3): 1024--39.

\bibitem[\citeproctext]{ref-Woo2023}
Woo, Byung-Deuk, Lindsey A Goldberg, and Frederick Solt. 2023. {``Public Gender Egalitarianism: A Dataset of Dynamic Comparative Public Opinion Toward Egalitarian Gender Roles in the Public Sphere.''} \emph{British Journal of Political Science} 53 (2): 766--75.

\bibitem[\citeproctext]{ref-Woo2024}
Woo, Byung-Deuk, Hyein Ko, Yuehong Cassandra Tai, Yue Hu, and Frederick Solt. Forthcoming. {``Public Support for Gay Rights Across Countries and over Time.''} \emph{Social Science Quarterly} Early View (Forthcoming).

\bibitem[\citeproctext]{ref-Wuttke2020}
Wuttke, Alexander, Christian Schimpf, and Harald Schoen. 2020. {``When the Whole Is Greater Than the Sum of Its Parts: On the Conceptualization and Measurement of Populist Attitudes and Other Multidimensional Constructs.''} \emph{American Political Science Review} 114 (2): 356--74.

\bibitem[\citeproctext]{ref-Zaki2022}
Zaki, Bishoy Louis, Francesco Nicoli, Ellen Wayenberg, and Bram Verschuere. 2022. {``In Trust We Trust: The Impact of Trust in Government on Excess Mortality During the COVID-19 Pandemic.''} \emph{Public Policy and Administration} 37 (2): 226--52.

\end{CSLReferences}

\pagebreak

\end{document}
